\documentclass[a4paper]{article}

\usepackage[a4paper, hmargin=25mm, vmargin=30mm, top=20mm]{geometry} % Use A4 paper and set margins

\usepackage{fancyhdr} % Customize the header and footer

\usepackage{lastpage} % Required for calculating the number of pages in the document

\usepackage{hyperref} % Colors for links, text and headings

\setcounter{secnumdepth}{0} % Suppress section numbering

\begin{document}


% title -----------------------------------------------------

\Large{\textbf{Rasmus Bengtsson - Curriculum vitae}}

\rule{\textwidth}{4px}



% misc info ---------------------------------------------
\normalsize{
\parbox{0.5\textwidth}{ % First block
\begin{tabbing} % Enables tabbing
\hspace{3cm} \= \hspace{4cm} \= \kill % Spacing within the block
{\bf Address:} \> Klintvägen 89,\\ % Address line 1
\> Luleå, 97335 \\ % Address line 2
{\bf Personnumer:} \> 20020831-4492 % Date of birth 
\end{tabbing}}
\hfill % Horizontal space between the two blocks
\parbox{0.5\textwidth}{ % Second block
\begin{tabbing} % Enables tabbing
\hspace{3cm} \= \hspace{4cm} \= \kill % Spacing within the block
{\bf Telefonnummer:} \> +46720229540 \\ % Mobile phone
{\bf Email:} \> \href{mailto:rasmus@beryllium-tech.eu}{rasmus@beryllium-tech.eu} \\ % Email address
{\bf LinkedIn:} \> \href{https://www.linkedin.com/in/rasmus-b-5b1019126/}{linkedin.com/in/rasmus-b-5b1019126/} \\ % Linkedin
{\bf Github:} \> \href{https://www.github.com/hackerbuddy229}{github.com/hackerbuddy229} % GITHUB
\end{tabbing}}}



\rule{0.5\textwidth}{2px}

% Aktuell resumé -----
\section{Aktuell resumé}

I nuläget så fyller jag mina dagar främst med att studera till min masters i datavetenskap
på Luleå Tekniska Universitet, medans jag pluggar i takten 100\% så har jag tagit på mig ett
stort engagemang i vårans studentkår där jag både jobbar för att drive vårat sammarbete med 
företagspartners framåt men också i Kårfullmäktige för att modernisera samt växa organisationens
omfång och allmäna verksamhet. Dem erfarenheterna jag samlar på mig är något jag värdesätter högt
då det både låter mig lösa komplexa problem på en stor skala men också då jag har en chans att
använda det för att se till att näst projekt jag tar på mig leder till ännu bättre resultat än det förra.
\\
\\
Min arbetstil är väldigt uppstrukturerad, detta är något jag tror är viktigt för att kunna 
arbeta effektift i ett team men lika viktigt är det att hitta en struktur som kommer
funkar för alla och för projecktet man jobbar på. Mycket tillfällen att få problemlösa är
vad som har drivigt mig till det yrket jag är intresserad av.

% Utbildning och certifieringar ----------
\section{Utbildning och certifieringar}
    \begin{itemize}
        \item{
            \textbf{2021-2026} Masters i datavetenskap - Luleå Tekniska Universitet 
        }

        \item{
            \textbf{2021} Introduktionskurs till Microsoft Azure - Volvos HR-data team 
        }

        \item{
            \textbf{2019-2021} Gymnasiexamen Teknik - NTI gymnasiet göteborg
        }

        \item{ % find correct information
            \textbf{2017} Elevskyddsombud - Arbetsmiljöverket
        }
    \end{itemize}

% Relevanta kunskaper
\section{Relevanta kunskaper}
    C\#, .NET/ASP.NET, Entity-framework, Rust, Javascript/Typescript, Java, REST/web-api,
    Enhetstestning, CI/CD, Containerization, Git, Linux, Latex, Microsoft Azure, Scrum,
    Kanban

% Erfarenheter
\section{Erfarenheter}

    \begin{itemize}
        \item {\textbf{2021-2026} Programrådsrepresentant datateknik - Teknologkåren vid Luleå Tekniska Universitet}
        \item {\textbf{2021/22} Företagskoordinator - Luleå Arbetsmarknadsvecka}
        \item {\textbf{2021/22} Kårfullmäktige Ledamot - Teknologkåren vid Luleå Tekniska Universitet}
        \item {\textbf{2021} Skrev ett ANN framework \textbf{Abyssal-AI} för att analysera Forex trender som gymnasiearbete}
        
    \end{itemize}

\end{document}